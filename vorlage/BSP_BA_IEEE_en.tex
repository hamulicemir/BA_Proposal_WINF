%
% FH Technikum Wien
% !TEX encoding = UTF-8 Unicode
%
% Erstellung von Master- und Bachelorarbeiten an der FH Technikum Wien mit Hilfe von LaTeX und der Klasse TWBOOK
%
% Um ein eigenes Dokument zu erstellen, müssen Sie folgendes ergänzen:
% 1) Mit \documentclass[..] einstellen
%    * Master- oder Bachelorarbeit
%    * Studiengang
%    * Sprache (english, german, ngerman)
%    * Zitationsstandard (Harvard, IEEE) (Standard: IEEE)
%    * Biber oder BibTeX als Literaturbackend (Biber, BibTeX) (Standard: Biber)
% 2) Deckblatt, Kurzfassung, etc. ausfüllen
% 3) und die Arbeit schreiben (die verwendeten Literaturquellen in Literatur.bib eintragen)
%
% Getestet mit TeXstudio mit Zeichenkodierung utf-8 (=ansinew/latin1) und TexLive unter Ubuntu
% Zu beachten ist, dass die Kodierung der Datei mit der Kodierung des paketes inputenc zusammen passt!
% Die Kodierung der Datei twbook.cls MUSS ANSI betragen!
% Bei der Verwendung von UTF8 muss nicht nur die Kodierung des Dokuments auf UTF8 gestellt sein, sondern auch die des BibTex-Files!
%
% Bugreports und Feedback bitte per E-Mail an latex@technikum-wien.at
%
% Version V2.24 von 2024-12-19 otrebski
%
\documentclass[BWI,Bachelor,english,IEEE]{twbook}%\documentclass[Bachelor,BMR,ngerman]{twbook}
\usepackage[utf8]{inputenc}
\usepackage[T1]{fontenc}

\addbibresource{Literatur.bib}
%% Definieren Sie hier bei Bedarf weitere Literaturdatenbanken

% Die nachfolgenden Pakete stellen sonst nicht benötigte Features zur Verfügung
\usepackage{blindtext}

%
% Einträge für Deckblatt, Kurzfassung, etc.
%
\title{Development and Evaluation of a Hybrid Approach
for Automated Error Detection and Classification in
LoRaWAN-Based IoT Data Pipelines Using Log Data}
\author{Emir Hamulic}
\studentnumber{wi23b168}
\supervisor{Rohatsch Lukas, MSc}

\place{Wien}
\kurzfassung{\blindtext}
\schlagworte{Schlagwort1, Schlagwort2, Schlagwort3, Schlagwort4}
\outline{\blindtext}
\keywords{Keyword1, Keyword2, Keyword3, Keyword4}
%\acknowledgements{\blindtext}

\begin{document}

\maketitle


% Hier können Sie Ihre KI-Tools dokumentieren. Diese werden automatisch in eine Tabelle integriert.
\aitoolentry{DeepL Translate}{Translation of an article in English}{Source (XXX), Chapter X on page X-X}
\aitoolentry{Chat GPT 4.0}{Grammar and Spelling}{"Please list issues with spelling and grammar in the following text: ..." Entire document}

%
% Hier beginnen die Verzeichnisse.
%
\clearpage
\printbibliography
\clearpage

% Das Abbildungsverzeichnis
\listoffigures
\clearpage

% Das Tabellenverzeichnis
\listoftables
\clearpage

% Das Verzeichnis über die verwendeten KI-Tools
\listaitools
\clearpage

\phantomsection
\addcontentsline{toc}{chapter}{\listacroname}
\chapter*{\listacroname}
\begin{acronym}[XXXXX]
    \acro{ABC}[ABC]{Alphabet}
    \acro{WWW}[WWW]{world wide web}
    \acro{ROFL}[ROFL]{Rolling on floor laughing}
\end{acronym}

%
% Hier beginnt der Anhang.
%
\clearpage
\appendix
\chapter{Anhang A}
\clearpage
\chapter{Anhang B}
\end{document}
